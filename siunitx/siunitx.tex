\documentclass{article}
\usepackage{amsmath, siunitx}
\usepackage[parfill]{parskip}

\sisetup{exponent-product= \cdot} % Multiplication symbol used in exponential form
\sisetup{separate-uncertainty = true} % write uncertainties as ± ∆a


\begin{document}
\section{Basic example}
A light bulb is connected to a voltage supply with \qty{12.2}{\V}. 
The current through the light bulb is \qty{54}{\milli\A}.
Calculate the power dissipated in the light bulb.

\begin{align}
    P &= \Delta V \cdot I \\
    &= \qty{12}{\V} \times \qty{54e-3}{\A} = \qty{0.648}{\W} \approx \qty{0.65}{W}
\end{align}

\section{Example with uncertainties}
For a resistor, the voltage and current are measured to be
 \qty{4.32+-0.01}{\V} and \qty{33.4+-0.2}{\milli\A}, respectively.
 Calculate the resistance with its uncertainty.

 \begin{align}
    R &= \frac{\Delta V}{I} \\
    &= \frac{\qty{4.32}{\V}}{\qty{33.4e-3}{\A}} = \qty{129.34}{\ohm}
 \end{align}

 \begin{align}
    \Delta R &= R_\textrm{max} - R \\
    &= \frac{\qty{4.33}{\V}}{\qty{33.2e-3}{\A}} - \qty{129.34}{\ohm} \\
    &= \qty{130.42}{\ohm} - \qty{129.34}{\ohm} = \qty{1.1}{\ohm}
 \end{align}

 \begin{align}
    R &= \qty{129.3+-1.1}{\ohm}
 \end{align}

\end{document}